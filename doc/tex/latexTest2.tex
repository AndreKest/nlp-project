\chapter{Verschiedene Ressourcen einbauen}

Diese werden in der \textbf{Datei} \texttt{literatur.bib} verwaltet. Die Einträge
dort können aus Zotero oder Jabref oder direkt aus Metadaten-Angaben/Downloads der Verlagswebseiten kommen.

Folgende Formel...  
$$
 \sum_{i=0}^N \frac{i}{i+1}
$$


Eine Matrix:

$$  
\left[\begin{array}{ll}
1.0	& 0.0 \\
0.0	& 1.0 \\
\end{array}
\right]
$$

\cite[Text als URL]{WillKurt.January312019}

Ein Buch mit Seitenzahl \cite[S. 34]{Balzert_etal2008}, ein Journal-Artikel \cite{Sperrvermerke_2015}, ein Proceedings-Beitrag \cite{Frantzi:1998:CMA:646631.696825}, sowie die Zotero-URL 
\cite{zoteroweb}. Bei den URL-Zitaten sollte immer auch ein \index{Datum} des Zugriffs angegeben werden (Zeile \texttt{lastchecked = ...}): 

\begin{lstlisting}[language=TeX]
@misc{zoteroweb,
  author = {Zotero},
  title = {Your personal research assistant},
  url = {https://zotero.org},
  lastchecked = {18.04.2020},
  year = 2020
}
\end{lstlisting}

Das Resultat ist auf der nächsten Seite zu sehen.
Die Verwendung des \texttt{listings}-Pakets ist in \cite{listingspkg} beschrieben.